% ============================================================================
%  MCKL/manual/tex/perf_random_distribution.tex
% ----------------------------------------------------------------------------
%  MCKL: Monte Carlo Kernel Library
% ----------------------------------------------------------------------------
%  Copyright (c) 2013-2016, Yan Zhou
%  All rights reserved.
%
%  Redistribution and use in source and binary forms, with or without
%  modification, are permitted provided that the following conditions are met:
%
%    Redistributions of source code must retain the above copyright notice,
%    this list of conditions and the following disclaimer.
%
%    Redistributions in binary form must reproduce the above copyright notice,
%    this list of conditions and the following disclaimer in the documentation
%    and/or other materials provided with the distribution.
%
%  THIS SOFTWARE IS PROVIDED BY THE COPYRIGHT HOLDERS AND CONTRIBUTORS "AS IS"
%  AND ANY EXPRESS OR IMPLIED WARRANTIES, INCLUDING, BUT NOT LIMITED TO, THE
%  IMPLIED WARRANTIES OF MERCHANTABILITY AND FITNESS FOR A PARTICULAR PURPOSE
%  ARE DISCLAIMED. IN NO EVENT SHALL THE COPYRIGHT HOLDER OR CONTRIBUTORS BE
%  LIABLE FOR ANY DIRECT, INDIRECT, INCIDENTAL, SPECIAL, EXEMPLARY, OR
%  CONSEQUENTIAL DAMAGES (INCLUDING, BUT NOT LIMITED TO, PROCUREMENT OF
%  SUBSTITUTE GOODS OR SERVICES; LOSS OF USE, DATA, OR PROFITS; OR BUSINESS
%  INTERRUPTION) HOWEVER CAUSED AND ON ANY THEORY OF LIABILITY, WHETHER IN
%  CONTRACT, STRICT LIABILITY, OR TORT (INCLUDING NEGLIGENCE OR OTHERWISE)
%  ARISING IN ANY WAY OUT OF THE USE OF THIS SOFTWARE, EVEN IF ADVISED OF THE
%  POSSIBILITY OF SUCH DAMAGE.
% ============================================================================

\chapter{Performance of distribution}
\label{chap:Performance of distribution}

The performance of distributions in sections~\ref{sec:Continuous distribution}
and~\ref{sec:Discrete distribution} are measured in single core cycles per
element (\cpe). The processor used is \cpuversion. The operating system is
\osversion. The compiler is \llvmversion. The \mkl library is used for
vectorized mathematical functions.

Four usage cases of the distributions are considered. First, if the
distribution is available in the standard library, we measure the following
case,
\begin{verbatim}
std::normal_distribution<double> rnorm_std(0, 1);
for (size_t i = 0; i = n; ++i)
    r[i] = rand(rng, rnorm_std);
\end{verbatim}
Second, we measure the performance of the library's implementation,
\begin{verbatim}
NormalDistribution<double> rnorm_mckl(0, 1);
for (size_t i = 0; i = n; ++i)
    r[i] = rand(rng, rnorm_mckl);
\end{verbatim}
The third is the vectorized performance,
\begin{verbatim}
rand(rng, rnorm_mckl, n, r.data());
\end{verbatim}
For all the three above, the \rng is |ARSx8| (section~\ref{sub:AES-NI
instructions based RNG}). The last is when there are \mkl routines for
generating random variates from the distribution directly (e.g., Normal
distribution) or indirectly (e.g., Student's $t$-distribution). In this case,
we use the \rng |MKL_MT2203| (see section~\ref{sec:MKL RNG}),
\begin{verbatim}
MKL_MT2203 rng_mkl;
rand(rng_mkl, rnorm_mckl, n, r.data());
\end{verbatim}
In all cases, we repeat the simulations 100 times, each time with the number of
elements $n$ chosen randomly between 5,000 and 10,000. The total number of
cycles of the 100 simulations are recorded, and then divided by the total
number of elements generated. This gives the performance measurement in \cpe.
This experiment is repeated ten times, and the best results are shown. The four
cases are labeled ``\std'', ``\mckl'', ``\batch'' and ``\mkl'', respectively,
in tables below.

\ExplSyntaxOn
\def\distributiontable#1#2#3{
  \file_if_exist:nT{tab/random_distribution_#1_llvm_sse2.tex}{
    \begin{table}
      \input{tab/random_distribution_#1_llvm_sse2}
      \caption{Performance~of~#2~(Nehalem,~sequential)}
      \label{tab:Performance~of~#3~(Nehalem,~sequential)}
    \end{table}
  }
  \file_if_exist:nT{tab/random_distribution_#1_llvm_sse2_p.tex}{
    \begin{table}
      \input{tab/random_distribution_#1_llvm_sse2_p}
      \caption{Performance~of~#2~(Nehalem,~parallel)}
      \label{tab:Performance~of~#3~(Nehalem,~parallel)}
    \end{table}
  }
  \file_if_exist:nT{tab/random_distribution_#1_llvm_avx2.tex}{
    \begin{table}
      \input{tab/random_distribution_#1_llvm_avx2}
      \caption{Performance~of~#2~(Haswell,~sequential)}
      \label{tab:Performance~of~#3~(Haswell,~sequential)}
    \end{table}
  }
  \file_if_exist:nT{tab/random_distribution_#1_llvm_avx2_p.tex}{
    \begin{table}
      \input{tab/random_distribution_#1_llvm_avx2_p}
      \caption{Performance~of~#2~(Haswell,~parallel)}
      \label{tab:Performance~of~#3~(Haswell,~parallel)}
    \end{table}
  }
  \clearpage
}
\ExplSyntaxOff

\distributiontable{inverse}{distributions using the inverse method}{%
distributions using the inverse method}
\distributiontable{beta}{Beta distribution}{Beta distribution}
\distributiontable{chisquared}{$\chi^2$ distribution}{chi-squared distribution}
\distributiontable{gamma}{Gamma distribution}{Gamma distribution}
\distributiontable{fisherf}{Fisher's $F$-distribution}{Fisher's F-distribution}
\distributiontable{normal}{Normal and related distributions}{%
Normal and related distributions}
\distributiontable{stable}{Stable distribution}{Stable distribution}
\distributiontable{studentt}{Student's $t$-distribution}{%
Student's t-distribution}
\distributiontable{int}{discrete distributions}{discrete distributions}
