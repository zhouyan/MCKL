% ============================================================================
%  MCKL/manual/tex/introduction.tex
% ----------------------------------------------------------------------------
%  MCKL: Monte Carlo Kernel Library
% ----------------------------------------------------------------------------
%  Copyright (c) 2013-2016, Yan Zhou
%  All rights reserved.
%
%  Redistribution and use in source and binary forms, with or without
%  modification, are permitted provided that the following conditions are met:
%
%    Redistributions of source code must retain the above copyright notice,
%    this list of conditions and the following disclaimer.
%
%    Redistributions in binary form must reproduce the above copyright notice,
%    this list of conditions and the following disclaimer in the documentation
%    and/or other materials provided with the distribution.
%
%  THIS SOFTWARE IS PROVIDED BY THE COPYRIGHT HOLDERS AND CONTRIBUTORS "AS IS"
%  AND ANY EXPRESS OR IMPLIED WARRANTIES, INCLUDING, BUT NOT LIMITED TO, THE
%  IMPLIED WARRANTIES OF MERCHANTABILITY AND FITNESS FOR A PARTICULAR PURPOSE
%  ARE DISCLAIMED. IN NO EVENT SHALL THE COPYRIGHT HOLDER OR CONTRIBUTORS BE
%  LIABLE FOR ANY DIRECT, INDIRECT, INCIDENTAL, SPECIAL, EXEMPLARY, OR
%  CONSEQUENTIAL DAMAGES (INCLUDING, BUT NOT LIMITED TO, PROCUREMENT OF
%  SUBSTITUTE GOODS OR SERVICES; LOSS OF USE, DATA, OR PROFITS; OR BUSINESS
%  INTERRUPTION) HOWEVER CAUSED AND ON ANY THEORY OF LIABILITY, WHETHER IN
%  CONTRACT, STRICT LIABILITY, OR TORT (INCLUDING NEGLIGENCE OR OTHERWISE)
%  ARISING IN ANY WAY OUT OF THE USE OF THIS SOFTWARE, EVEN IF ADVISED OF THE
%  POSSIBILITY OF SUCH DAMAGE.
% ============================================================================

\chapter{Introduction}
\label{chap:Introduction}

\section{Monte Carlo algorithms}
\label{sec:Monte Carlo algorithms}

\section{Using the library}
\label{sec:Using the library}

\subsection{Installation}
\label{sub:Installation}

\subsection{Documents}
\label{sub:Documents}

\section{Organization of headers}
\label{sec:Organization of headers}

\mckl is a header-only library. All headers files are under the \verb|mckl|
directory. To include all functionalities,
\begin{Verbatim}
#include <mckl/mckl.hpp>
\end{Verbatim}
There are a few other headers that include a subset of functions of \mckl, each
documented in a subsequent chapter in this manual. They are listed in
Table~\ref{tab:headers}. These headers does not define anything other than
including other headers that each implement a specific feature. If only one a
few features are needed, then one can include only headers that implement those
features to save compilation time. For example,
\begin{Verbatim}
#include <mckl/random/threefry.hpp>
\end{Verbatim}
includes only the header that implements \rng engines based on the Threefry
algorithm in~\cite{Salmon:2011um} (see Section~\ref{sub:Threefry}). See the
reference manual\footnote{\url{http://zhouyan.github.io/MCKLDoc/master}} for
the header file for each class and function defined in \mckl.

\begin{table}
  \begin{tabularx}{\textwidth}{LL}
    \toprule
    Header & Documents \\
    \midrule
    \verb|mckl/core.hpp|     & Chapter~\ref{chap:Core concepts}             \\
    \verb|mckl/smp.hpp|      & Chapter~\ref{chap:Symmetric multiprocessing} \\
    \verb|mckl/resample.hpp| & Chapter~\ref{chap:Resampling}                \\
    \verb|mckl/math.hpp|     & Chapter~\ref{chap:Mathemtical functions}     \\
    \verb|mckl/random.hpp|   & Chapter~\ref{chap:Random number generating}  \\
    \verb|mckl/random/rng.hpp|
    & Sections~\ref{sec:Counter-based RNG} to~\ref{sec:MKL RNG} \\
    \verb|mckl/random/distribution.hpp|
    & Section~\ref{sec:Distribution} \\
    \verb|mckl/random/test.hpp|
    & Section~\ref{sec:Randomness tests} \\
    \verb|mckl/utility|      & Chapter~\ref{chap:Utilities}                 \\
    \bottomrule
  \end{tabularx}
  \caption{Top-level headers}
  \label{tab:headers}
\end{table}
