% ============================================================================
%  MCKL/manual/tex/introduction.tex
% ----------------------------------------------------------------------------
%                          MCKL: Monte Carlo Kernel Library
% ----------------------------------------------------------------------------
%  Copyright (c) 2013-2016, Yan Zhou
%  All rights reserved.
%
%  Redistribution and use in source and binary forms, with or without
%  modification, are permitted provided that the following conditions are met:
%
%    Redistributions of source code must retain the above copyright notice,
%    this list of conditions and the following disclaimer.
%
%    Redistributions in binary form must reproduce the above copyright notice,
%    this list of conditions and the following disclaimer in the documentation
%    and/or other materials provided with the distribution.
%
%  THIS SOFTWARE IS PROVIDED BY THE COPYRIGHT HOLDERS AND CONTRIBUTORS "AS IS"
%  AND ANY EXPRESS OR IMPLIED WARRANTIES, INCLUDING, BUT NOT LIMITED TO, THE
%  IMPLIED WARRANTIES OF MERCHANTABILITY AND FITNESS FOR A PARTICULAR PURPOSE
%  ARE DISCLAIMED. IN NO EVENT SHALL THE COPYRIGHT HOLDER OR CONTRIBUTORS BE
%  LIABLE FOR ANY DIRECT, INDIRECT, INCIDENTAL, SPECIAL, EXEMPLARY, OR
%  CONSEQUENTIAL DAMAGES (INCLUDING, BUT NOT LIMITED TO, PROCUREMENT OF
%  SUBSTITUTE GOODS OR SERVICES; LOSS OF USE, DATA, OR PROFITS; OR BUSINESS
%  INTERRUPTION) HOWEVER CAUSED AND ON ANY THEORY OF LIABILITY, WHETHER IN
%  CONTRACT, STRICT LIABILITY, OR TORT (INCLUDING NEGLIGENCE OR OTHERWISE)
%  ARISING IN ANY WAY OUT OF THE USE OF THIS SOFTWARE, EVEN IF ADVISED OF THE
%  POSSIBILITY OF SUCH DAMAGE.
% ============================================================================

\chapter{Introduction}
\label{chap:Introduction}

In this chapter, we introduce the basic structure of Monte Carlo algorithms
that can be implemented with this library. At a high level, almost all
algorithms can be described as the following, from an implementation
perspective. Let $\{E_t\}_{t\ge0}$ be a sequence of state spaces, and
$\{X_t^i\}_{i=1}^{N_t} \in E_t^{N_t}$. A Monte Carlo algorithm iteratively
applies some kernel $M_t:\prod_{k=0}^{t-1}E_k^{N_k}\to\prod_{k=0}^tE_k^{N_k}$,
that draws $\{X_k^{1:N_k}\}_{k=1}^t$ given $\{X_k^{1:N_k}\}_{k=0}^{t-1}$
according to some probability law.

At first this may seem a very strange way to describe Monte Carlo algorithms.
But later it will become clear that, we can implement many always by just
defining proper kernels $M_t$.

\section{Using the library}
\label{sec:Using the library}

\subsection{Installation}
\label{sub:Installation}

\subsection{Documents}
\label{sub:Documents}

\section{Organization of headers}
\label{sec:Organization of headers}
